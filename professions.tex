\chapter{Professions}

\begin{table}[H]
	\centering
	\begin{tabular}{lll}
	\toprule
		\textbf{Profession} & \textbf{\emph{Traducci\'on}} & \textbf{Phonetic Pronunciation}\\
	\midrule
		server/waiter & \emph{mesero} & sehr-vehr/wey-tehr \\
		cashier & \emph{cajero} & ca*-shee-uhr \\
		salesperson & \emph{vendedor} & seyls-pehr-suhn \\
		mechanic & \emph{mecánico} & meh-ka*-nihk \\
		custodian & \emph{conserje} & kuh-stoh-dee-ehn \\
		customer & \emph{cliente} & kuh-stuh-muhr \\
		receptionist & \emph{recepcionista} & rih-sehp-shuhn-ihst \\
		bus driver & \emph{conductor de autobús} & buhs drah-ee-vuhr \\
		homemaker & \emph{ama de casa} & hohm-mey-kuhr \\
		plumber & \emph{plomero} & pluhm-buhr\\
		painter & \emph{pintor} & peyn-tuhr \\
		truck driver & \emph{camionero} & truhk drah-ee-vuhr \\
		teacher's aide & \emph{ayudante del profesor/auxiliar} & tee-chuhrz eyd \\
	\bottomrule
	\end{tabular}
	\caption{Palabras de vocabulario: professions}
\end{table}

\section{Hacer preguntas}%
\label{sec:Hacer preguntas}

En ingl\'es, las preguntas que se pueden contestar con «s\'i» o «no» comienzan con una conjugaci\'on de un verbo auxiliar, t\'ipicamente, el verbo \emph{to do} (hacer, en unos contextos).
En otros casos, se usa el verbo \emph{to be} (ser/estar), y, con menos frecuencia, un verbo modal.
Por ahora, no vamos a preocuparnos por los verbos modales.\\

Repasemos la conjugaci\'on del verbo \emph{to do}:

\begin{table}[H]
	\centering
	\begin{tabular}{lll}
	\toprule
		\textbf{Pronoun} & \textbf{Present tense conjugation} & \textbf{Past tense conjugation}\\
	\midrule
		I & do & did\\
		You & do & did\\
		She/He/It & does & did \\
		We & do & did\\
		They & do & did\\
	\bottomrule
	\end{tabular}
	\caption{``to do'' Conjugation}
\end{table}

Esas preguntas siguen la forma siguiente:
<conjugaci\'on correcta del verbo auxiliar> <nombre o pronombre personal> <el resto de la frase>. \\

Se puede contestar de manera afirmativa con
«Yes, <nombre o pronombre personal> <conjugación correspondiente del verbo auxiliar>
Para contestar de manera negativa, hay 2 opciones:
\begin{itemize}
	\item «No, <nombre o pronombre personal> <contracción del verbo auxiliar con \emph{not}>»
	\item «No, <nombre o pronombre personal> <conjugación correspondiente del verbo auxiliar> not»
\end{itemize}

Veamos unos ejemplos:



\begin{table}[H]
	\rowcolors{2}{blue!15}{white}
	\centering
	\begin{tabular}{lp{7cm}p{8cm}}
		\toprule
			& \textbf{English} & \textbf{\emph{Traducci\'on}} \\
		\midrule
			1 & Does he sell clothes? & \emph{\textquestiondown Él vende ropa?} \\
				& Yes, he does/No, he doesn't. & \emph{Sí, vende ropa/No, no vende ropa.} \\
			2 & Is he a painter? & \emph{\textquestiondown Él es pintor?} \\
				& Yes, he is/No, he isn't. & \emph{Sí, lo es/No, no lo es.} \\
			3 & Do they serve food? & \emph{\textquestiondown Sirven comida?} \\
				& Yes, they do/No, they don't. & \emph{Sí, sirven comida/ No, no sirven comida} \\
			4 & Are you a receptionist? & \emph{\textquestiondown Eres recepcionista?} \\
				& Yes, I am/No, I am not. & \emph{Sí, lo soy/No, no lo soy} \\
		\bottomrule
	\end{tabular}
\end{table}

Las preguntas negativas siguien la forma siguiente:
\begin{itemize}
	\item <conjugación correcta del verbo auxiliar> <nombre o pronombre personal> not <el resto de la frase>
	\item <contracción del verbo auxiliar con \emph{not}> <nombre o pronombre personal> <el resto de la frase> 
\end{itemize}
Estas preguntas se contestan en la misma manera.

\begin{table}[H]
	\rowcolors{2}{blue!15}{white}
	\centering
	\begin{tabular}{lp{7cm}p{8cm}}
		\toprule
			& \textbf{English} & \textbf{\emph{Traducci\'on}} \\
		\midrule
			1 & Does he not he sell clothes? & \emph{\textquestiondown Él no vende ropa?} \\
				& Yes, he does/No, he doesn't. & \emph{Sí, vende ropa/No, no vende ropa.} \\
			2 & Isn't he a painter? & \emph{\textquestiondown Él no es pintor?} \\
				& Yes, he is/No, he isn't. & \emph{Sí, lo es/No, no lo es.} \\
			3 & Don't they serve food? & \emph{\textquestiondown No siven comida?} \\
				& Yes, they do/No, they don't. & \emph{Sí, sirven comida/ No, no sirven comida} \\
			4 & Are you not a receptionist? & \emph{\textquestiondown No eres recepcionista?} \\
				& Yes, I am/No, I am not. & \emph{Sí, lo soy/No, no lo soy} \\
		\bottomrule
	\end{tabular}
\end{table}

\begin{conf}{Un patr\'on}
	Un patr\'on que les puede ayudar con unos sustantivos que sirven para describir otros sustantivos:
	\begin{itemize}
		\item leche de \underline{chocolate} \arr \emph{\underline{chocolate} milk}
		\item clase de \underline{matemáticas} \arr \emph{\underline{mathematics/math} class}
		\item anillo de \underline{oro} \arr \emph{\underline{gold} ring}
	\end{itemize}

	\textquestiondown Pueden ver el patr\'on?
	En esas frases, las palabras subrayadas describen el sustantivo.
	En ingl\'es, no se incluye la preposici\'on «de», y la palabra que describe el otro sustantivo está enfrente de
	la palabra que se describe.
\end{conf}

\begin{conf}{Verbos}
	Comencemos con la siguiente frases:
	\begin{enumerate}
		\item ``\textcolor{blue}{Does} she \textcolor{red}{do} her homework?'' \arr
			\emph{\textquestiondown Ella hace su tarea?}
		\item ``\textcolor{blue}{Does} he \textcolor{red}{water} water the plants?''
			\arr \emph{\textquestiondown Él riega las plantas?}
	\end{enumerate}

	Como ya han visto, para la gran mayoría de verbos, hay solamente 2 conjugaciones en el presente:
	\begin{enumerate}
		\item una forma que se utiliza con \emph{he/she}
		\item y otra forma que se utiliza con los otros pronombres.
	\end{enumerate}
	Les doy un ejemplo, ``to water'':

	\begin{table}[H]
		\centering
		\begin{tabular}{lll}
		\toprule
			\textbf{Pronoun} & \textbf{Conjugation}\\
		\midrule
			I & water \\
			You & water \\
			She/He/It & waters \\
			We & water \\
			They & water \\
		\bottomrule
		\end{tabular}
		\caption{``to water'' Conjugation}
	\end{table}

	Como se puede ver, la forma ``waters'' se usa con las formas de \emph{he/she}, y ``water''
	en los otros casos.
	En las frases interrogativas, siempre se usa la segunda conjugación con el segundo verbo, el verbo que indica
	la acción. \\ \\

	Esto se puede ver con los ejemplos arriba. Los sujetos en los ejemplos son \emph{she}
	y \emph{he} respectivamente. El primer verbo (el verbo auxiliar, indicado en azul) sigue la conjugación normal.
	El segundo verbo (el que indica la acción, indicado en rojo), se conjuga en la otra manera (no de la forma \emph{he/she},
	aunque el sujeto sea \emph{he/she}).
\end{conf}
