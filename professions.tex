\chapter{Professions}

\begin{table}[H]
	\centering
	\begin{tabular}{lll}
	\toprule
		\textbf{Profession} & \textbf{\ita{Traducci\'on}} & \textbf{Phonetic Pronunciation}\\
	\midrule
		server/waiter & \ita{mesero} & sehr-vehr/wey-tehr \\
		cashier & \ita{cajero} & ca*-shee-uhr \\
		salesperson & \ita{vendedor} & seyls-pehr-suhn \\
		mechanic & \ita{mecánico} & meh-ka*-nihk \\
		custodian & \ita{conserje} & kuh-stoh-dee-ehn \\
		customer & \ita{cliente} & kuh-stuh-muhr \\
		receptionist & \ita{recepcionista} & rih-sehp-shuhn-ihst \\
		bus driver & \ita{conductor de autobús} & buhs drah-ee-vuhr \\
		homemaker & \ita{ama de casa} & hohm-mey-kuhr \\
		plumber & \ita{plomero} & pluhm-buhr\\
		painter & \ita{pintor} & peyn-tuhr \\
		truck driver & \ita{camionero} & truhk drah-ee-vuhr \\
		teacher's aide & \ita{ayudante del profesor/auxiliar} & tee-chuhrz eyd \\
	\bottomrule
	\end{tabular}
	\caption{Palabras de vocabulario: professions}
\end{table}

\section{Hacer preguntas}%
\label{sec:Hacer preguntas}

En ingl\'es, las preguntas que se pueden contestar con «s\'i» o «no» comienzan con una conjugaci\'on de un verbo auxiliar, t\'ipicamente, el verbo \ita{to do} (hacer, en unos contextos).
En otros casos, se usa el verbo \ita{to be} (ser/estar), y, con menos frecuencia, un verbo modal.
Por ahora, no vamos a preocuparnos por los verbos modales.\\

Repasemos la conjugaci\'on del verbo \ita{to do}:

\begin{table}[H]
	\centering
	\begin{tabular}{lll}
	\toprule
		\textbf{Pronoun} & \textbf{Present tense conjugation} & \textbf{Past tense conjugation}\\
	\midrule
		I & do & did\\
		You & do & did\\
		She/He/It & does & did \\
		We & do & did\\
		They & do & did\\
	\bottomrule
	\end{tabular}
	\caption{``to do'' Conjugation}
\end{table}

Las preguntas siguen la forma siguiente: <conjugaci\'on correcta del verbo auxiliar> + <el pronombre personal> + <el resto de la frase>. \\

Veamos unos ejemplos:


\begin{conf}{Un patr\'on}
	Un patr\'on que les puede ayudar con unos sustantivos que sirven para describir otros sustantivos:
	\begin{itemize}
		\item leche de \underline{chocolate} \arr \ita{\underline{chocolate} milk}
		\item clase de \underline{matemáticas} \arr \ita{\underline{mathematics/math} class}
		\item anillo de \underline{oro} \arr \ita{\underline{gold} ring}
	\end{itemize}

	\textquestiondown Pueden ver el patr\'on?
	En esas frases, las palabras subrayadas describen el adjetivo.
	En ingl\'es, no se incluye la preposici\'on «de», y la palabra que describe el otro sustantivo está enfrente de
	la palabra que se describe.
\end{conf}
