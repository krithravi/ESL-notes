\chapter{Time}

\section{Telling time}

En este cap\'itulo, vamos a aprender a contestar la pregunta «\textquestiondown Qu\'e hora es?» (\ita{What time is it?}).
Espero que se diviertan. :) \\

La forma b\'asica de contestar depende si la hora es exacta, o si no lo es.
Con las horas exactas, se contesta con \ita{el n\'umero cardinal de la hora} + \ita{o'clock}.
Por ejemplo, la frase «Son las diez en punto», se traduce como \ita{It is ten o'clock.}
Sin embargo, si la hora no es exacta, s\'olo se contesta con
\ita{el n\'umero cardinal de la hora} + \ita{el n\'umero cardinal de los minutos}; no se agrega ese \ita{o'clock}.
Por ejemplo, la frase «Son las once y treinticuatro», se contesta con \ita{It is eleven thirty-four.}
\footnote{Estas horas se mencionan en la canci\'on «Otra noche en Miami» por el rapero Bad Bunny.} \\

Vamos a ver unos ejemplos:
\begin{itemize}
	\item Son las cuatro. \arr \ita{It is four o'clock.}
	\item Son las cuatro y cuarenta. \arr \ita{It is four forty.}
	\item Es la una. \arr \ita{It is one o'clock.}
	\item Es la una y media. \arr \ita{It is one thirty.}
	\item Son las nueve. \arr \ita{It is nine o'clock.}
	\item Son las nueve y veinte. \arr \ita {It is nine twenty.}
\end{itemize}

En muchos pa\'ises angloparlantes, como los Estados Unidos, no se suele utilizar un sistema horario de 24 horas.
Por eso, se usan las expresiones \ita{AM} y \ita{PM} para indicar la mañana o la tarde/noche respectivamente.
Cuando usamos \ita{AM} o \ita{PM} con las horas exactas, jam\'as se incluye ese \ita{o'clock}. \ita{AM} se pronuncia como \ita{ey-ehm} y \ita{PM} se pronuncia como \ita{pee-ehm}.
Tambi\'en, se puede usar:
\begin{itemize}
	\item in the morning \arr \ita{de la ma\~nana}
	\item in the afternoon/evening \arr \ita{de la tarde}
	\item in the night \arr \ita{de la noche}
\end{itemize}
Cuando usamos las expresiones \ita{in the morning}, \ita{in the afternoon/evening}, y \ita{in the night} con
horas exactas, se puede incluir \ita{o'clock}. Por ejemplo, la frase «Son las 3 de la ma\~nana» se puede traducir
como \ita{It is three in the morning} o \ita{It is three o'clock in the morning}; son equivalantes. \\

Vean aqu\'i unas frases con sus traducciones equivalantes.:

\begin{table}[H]
	\centering
	\begin{tabular}{ll}
		\toprule
		\textbf{Frase} & \textbf{Traducci\'on} \\
		\midrule
		Son las 6 de la ma\~nana & It is six AM \\
			& It is six o'clock in the morning \\
			& It is six in the morning \\
		\hline
		Son las 4 de la tarde & It is four PM \\
			& It is four in the afternoon \\
			& It is four o'clock in the afternoon \\
			& It is four in the evening \\
			& It is four o'clock in the evening \\
		\hline
		Son las 10 de la noche & It is ten PM \\
			& It is ten in the night. \\
			& It is ten o'clock in the night \\
		\hline
		Son las 9:38 de la ma\~nana & It is nine thirty-eight AM \\
			& It is nine thirty-eight in the morning. \\
		\hline
		Son las 5:28 de la tarde & It is five twenty-eight PM \\
			& It is five twenty-eight in the afternoon \\
			& It is five twenty-eight in the evening \\
		\bottomrule
	\end{tabular}
	\caption{La hora}
\end{table}


\subsection{Otras maneras de expresar la hora}

En espa\~nol, en vez de decir 12 de la ma\~nana, se dice medianoche.
Se dice mediod\'ia en vez de decir 12 de la tarde.
En ingl\'es, la medianoche es \ita{midnight}, y el mediod\'ia es \ita{noon}. \\

Creo que la manera m\'as f\'acil de explicar lo siguiente es por otra tabla.
La sangre de las ranas de bosque se congela durante los inviernos.
\footnote{S\'i, \textexclamdown es verdad! \textquestiondown Est\'an prestando atenci\'on?}

\begin{table}[H]
	\centering
	\begin{tabular}{ll}
		\toprule
		\textbf{Frase} & \textbf{Traducci\'on} \\
		\midrule
		Son las ocho y cuarto & It is quarter past eight. \\
		Son las seis y media & It is half past eight \\
		Son las nueve y diez & It is ten past nine. \\
		\hline
		Faltan quince para las ocho & It is quarter to eight. \\
		Faltan cinco para las doce & It's five to twelve. \\
		Faltan diez para las tres & It's ten to three. \\
		\bottomrule
	\end{tabular}
	\caption{La hora - otras expresiones}
\end{table}

\ita{Quarter} significa «cuarto», o «cuarta hora», que es 15 minutos.
\ita{Half} significa «mitad», y la mitad de una hora es 30 minutos.
Recuerden que en la palabra \ita{half}, la «l» no se pronuncia.

\section{Events}

Para decir que alg\'un evento toma lugar a alguna hora, se utiliza la preposici\'on \textbf{\ita{at}}. \\

Empecemos con un poco de vocabulario:
\begin{table}[H]
	\centering
	\begin{tabular}{lll}
		\toprule
		\textbf{Event} & \textbf{\ita{Evento}} & \textbf{Phonetic Pronunciation}\\
		\midrule
		appintment & \ita{cita} & uhp-oint-mehnt \\
		movie & \ita{pel\'icula} & moo-vee \\
		class & \ita{clase} & cla*s \\
		meeting & \ita{reuni\'on} & mee-tihng \\
		party & \ita{fiesta} & pahrtee \\
		TV show & \ita{serie/programa} & tee-vee shoh-oo \\
		soap opera & \ita{telenovela} & sohp ah-pehr-ah \\
		\bottomrule
	\end{tabular}
	\caption{Palabras de vocabulario - eventos}
\end{table}


Ahora, vamos a ver unas oraciones con sus traducciones.
Recuerden que para algunas oraciones, hay varias traducciones correctas;
s\'olo he incluido una respuesta possible.
Otra vez, recuerden que se usa la preposici\'on \ita{at} con las expresiones siguientes,
porque estamos hablando de la hora a la cual algo va a pasar.

\begin{itemize}
	\item \ita{La pel\'icula est\'a a las siete en punto}
		\arr The movie is at seven o'clock.
	\item \ita{Tengo una cita a las cuatro}
		\arr I have an appointment at four o'clock.
	\item \ita{Voy a clase a las siete de la noche.}
		\arr I am going to class at seven in the night.
	\item \ita{La novela comienza a las seis de la tarde.}
		\arr The soap opera begins at six o'clock in the evening.
	\item \ita{Tomo un caf\'e a las cinco y media de la ma\~nana.}
		\arr I drink coffee at half past five in the morning.
	\item \ita{Llego al trabajo a las ocho de la ma\~nana.}
		\arr I get to work at eight AM.
	\item \ita{Tenemos una fiesta a las nueve y cuarto de la noche.}
		\arr We have a party at quarter past nine in the night.
	\item \ita{Mi clase de matem\'aticas comienza a las nueve de la ma\~nana}
		\arr My math class begins at 9 AM.
\end{itemize}

\section{Dates and Days}

Las fechas siguen la siguiente forma: <mes> <número ordinal del d\'ia>.
He puesto aqu\'i una lista de algunos n\'umeros ordinales con sus traducciones y formas cortas.

\begin{table}[H]
	\centering
	\begin{tabular}{lll}
		\toprule
		\textbf{Ordinal number} & \textbf{Traducci\'on} & \textbf{Forma corta}\\
		\midrule
		first & \ita{primero} & 1\textsuperscript{st} \\
		second & \ita{segundo} & 2\textsuperscript{nd} \\
		third & \ita{tercero} & 3\textsuperscript{rd} \\
		fourth & \ita{cuarto} & 4\textsuperscript{th} \\
		fifth & \ita{quinto} & 5\textsuperscript{th} \\
		. & . & . \\
		. & . & . \\
		. & . & . \\
		twentieth & \ita{vig\'esimo} & 20\textsuperscript{th} \\
		twenty-first & \ita{vig\'esimo primero} & 21\textsuperscript{st} \\
		twenty-second & \ita{vig\'esimo segundo} & 22\textsuperscript{nd} \\
		twenty-third & \ita{vig\'esimo tercero} & 23\textsuperscript{rd} \\
		twenty-fourth & \ita{vig\'esimo cuarto} & 24\textsuperscript{th} \\
		. & . & . \\
		. & . & . \\
		. & . & . \\
		thirtieth & \ita{trig\'esimo} & 30\textsuperscript{th} \\
		. & . & . \\
		. & . & . \\
		. & . & . \\
		hundredth & \ita{cent\'esimo} & 100\textsuperscript{th} \\
		\bottomrule
	\end{tabular}
	\caption{Palabras de vocabulario - ordinal numbers}
\end{table}


Por eso, las expresiones siguientes son correctas:
\begin{itemize}
	\item \ita{el siete de marzo} \arr March 7\textsuperscript{th}
	\item \ita{el primero de enero} \arr January 1\textsuperscript{st}
	\item \ita{el cuatro de octubre} \arr October 4\textsuperscript{th}
\end{itemize}

Se usa la preposici\'on \textbf{\ita{on}} con los d\'ias y las fechas.
Sin embargo, si s\'olo se incluye el mes, la preposici\'on que se usa es \textbf{\ita{in}}. \\

Con unos ejemplos, las cosas se aclarar\'an:
\begin{itemize}
	\item \ita{La navidad es en diciembre}
		\arr Christmas is \textbf{in} December.
	\item \ita{La navidad es el 25 de diciembre}
		\arr Christmas is \textbf{on} December 25\textsuperscript{th}.
	\item \ita{Alejandro naci\'o en agosto}
		\arr Alejandro was born \textbf{in} August.
	\item \ita{Alejandro naci\'o el 4 de agosto}
		\arr Alejandro was born \textbf{on} August 4\textsuperscript{th}.
	\item \ita{El viernes, tengo que ir al dentista}
		\arr \textbf{On} Friday, I have to go to the dentist.
	\item \ita{Mark tiene que ir a clase los viernes.}
		\arr Mark has to go to class \textbf{on} Fridays.
	\item \ita{Tengo una cita el viernes a las cinco de la tarde.}
		\arr I have an appointment \textbf{on} Friday \textbf{at} five PM.
	\item \ita{Alan va al teatro el domingo a las seis de la tarde.}
		\arr Alan is going to the movie theater \textbf{on} Sunday at six PM.
	\item \ita{La fiesta de Jenny toma lugar el s\'abado a las ocho y media.}
		\arr Jenny's party takes place \textbf{on} Saturday \textbf{at} half past eight.
\end{itemize}


Si queremos incluir el nombre del d\'ia, el mes y el d\'ia, se sigue la siguiente estructura:
<día>, <mes> <número ordinal del d\'ia>.
Por ejemplo, se puede decir \ita{Saturday, November 20\textsuperscript{th}}.
La traducci\'on de la frase «Tengo una fiesta el d\'ia 5 de abril el lunes a las seis de la tarde» ser\'ia
\ita{I have a party on Monday, April 5\textsuperscript{th} at 6 PM}.
Vean aqu\'i que se usa \ita{on} antes del d\'ia, pero \ita{at} antes de la hora. \\

