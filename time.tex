\chapter{Time}

\section{Telling time}

En este cap\'itulo, vamos a aprender a contestar la pregunta «\textquestiondown Qu\'e hora es?» (\ita{What time is it?}).
Espero que se diviertan. :) \\

La forma b\'asica de contestar depende si la hora es exacta, o si no lo es.
Con las horas exactas, se contesta con \ita{el n\'umero cardinal de la hora} + \ita{o'clock}.
Por ejemplo, la frase «Son las diez en punto», se traduce como \ita{It is ten o'clock.}
Sin embargo, si la hora no es exacta, s\'olo se contesta con \ita{el n\'umero cardinal de la hora} + \ita{el n\'umero cardina de los minutos}; no se agrega ese \ita{o'clock}.
Por ejemplo, la frase «Son las once y treinticuatro», se contesta con \ita{It is eleven thirty-four.}\footnote{Estas horas se mencionan en la canci\'on «Otra noche en Miami» por el rapero Bad Bunny.} \\

Vamos a ver unos ejemplos:
\begin{itemize}
	\item Son las cuatro. \arr \ita{It is four o'clock.}
	\item Son las cuatro y cuarenta. \arr \ita{It is four forty.}
	\item Es la una. \arr \ita{It is one o'clock.}
	\item Es la una y media. \arr \ita{It is one thirty.}
	\item Son las nueve. \arr \ita{It is nine o'clock.}
	\item Son las nueve y veinte. \arr \ita {It is nine twenty.}
\end{itemize}

En muchos pa\'ises angloparlantes, como los Estados Unidos, no se suele utilizar un sistema horario de 24 horas.
Por eso, se usan las expresiones \ita{AM} y \ita{PM} para indicar la mañana o la tarde/noche respectivamente.
Cuando usamos \ita{AM} o \ita{PM} con las horas exactas, jam\'as se incluye ese \ita{o'clock}.
Tambi\'en, se puede usar:
\begin{itemize}
	\item \ita{in the morning} \arr \ita{de la ma\~nana}
	\item \ita{in the afternoon/evening} \arr \ita{de la tarde}
	\item \ita{in the night} \arr \ita{de la noche}
\end{itemize}
Cuando usamos las expresiones \ita{in the morning}, \ita{in the afternoon/evening}, y \ita{in the night} con
horas exactas, se puede incluir \ita{o'clock}. Por ejemplo, la frase «Son las 3 de la ma\~nana» se puede traducir
como \ita{It is three in the morning} o \ita{It is three o'clock in the morning}; son equivalantes. \\

Vean aqu\'i unas frases con sus traducciones equivalantes.:

\begin{table}[H]
	\centering
	\begin{tabular}{ll}
		\toprule
		\textbf{Frase} & \textbf{Traducci\'on} \\
		\midrule
		Son las 6 de la ma\~nana & \ita{It is six AM} \\
			& It is six o'clock in the morning \\
			& It is six in the morning \\
		Son las 4 de la tarde & \ita{It is four PM} \\
			& It is four in the afternoon \\
			& It is four o'clock in the afternoon \\
			& It is four in the evening \\
			& It is four o'clock in the evening \\
		Son las 10 de la noche & \ita{It is ten PM} \\
			& \ita{It is ten in the night.} \\
			& \ita{It is ten o'clock in the night} \\
		Son las 9:30 de la ma\~nana & \ita{It is nine thirty AM} \\
		\bottomrule
	\end{tabular}
	\caption{La hora}
\end{table}


\subsection{Otras maneras de expresar la hora}

En espa\~nol, se puede decir 12 AM, o medianoche, o 12 PM, mediod\'ia.
En ingl\'es, se dice 

\section{Events}


\section{Dates and Days}
