\chapter{Countries}

\begin{itemize}
	\item Countries, nationalities and languages are \textbf{always}
		capitalized in English.
	\item In this section, the hyphens (-) don't mean anything.
		No indican las sílabas, ni nada.
\end{itemize}

\section{Countries}


\begin{table}[H]
	\center
	\begin{tabular}{lll}
	\toprule
	\textbf{Country} & \textbf{\ita{Pa\'is}} & \textbf{Phonetic Pronunciation} \\
	\midrule
	United States (USA) & \ita{Estados Unidos (EEUU)} & Yoo-nah-i-tehd Steyts \\
	Mexico & \ita{M\'exico} & Mehk-sih-coh \\
	Brazil & \ita{Brasil} & Bruh-zil \\
	Haiti & \ita{Hait\'i} & Hey-tee \\
	Russia & \ita{Rusia} & Ruh-shah \\
	Somalia & \ita{Somalia} & Soh-mah-lee-uh \\
	Vietnam & \ita{Vietnam} & Vee-eht-nahm \\
	China & \ita{China} & Chah-ee-nah \\
	Colombia & \ita{Colombia} & Coh-luhm-bee-ah \\
	Venezuela & \ita{Venezuela} & Veh-neh-zoo-eh-la \\
	Honduras & \ita{Honduras} & Hawn-der-ehs \\
	Guatemala & \ita{Guatemala} & Gwa-tih-mah-luh \\
	Cuba & \ita{Cuba} & Kyoo-bah \\
	\bottomrule
	\end{tabular}
	\caption{Countries}
\end{table}

\section{Nationalities}
\begin{table}[H]
	\center
	\begin{tabular}{lll}
	\toprule
	\textbf{Nationality} & \textbf{\ita{Nacionalidad}} & \textbf{Phonetic Pronunciation} \\
	\midrule
	American & \ita{estadounidense} & Uh-meh-rih-kehn \\
	Mexican & \ita{mexicano} & Mehk-sih-kehn \\
	Brazilian & \ita{brasile\~no} & Bruh-zil-yen \\
	Haitian & \ita{haitiano} & Hey-shehn \\
	Russian & \ita{ruso} & Ruh-shee-ehn \\
	Somalian & \ita{somalí} & Soh-mah-lee-uhn \\
	Vietnamese & \ita{vietnamita} & Vee-eht-nah-mees \\
	Chinese & \ita{chino} & Chah-ee-nees\\
	Colombian & \ita{colombiano} & Coh-luhm-bee-yehn \\
	Venezuelan & \ita{venezolano} & Veh-neh-zoo-eh-lehn \\
	Honduran & \ita{hondure\~no} & Hawn-der-ehn \\
	Guatemalan & \ita{guatemalteco} & Gwa-tih-mahl-teh-kehn \\
	\bottomrule
	\end{tabular}
	\caption{Nationalities}
\end{table}

\section{Languages}
\begin{table}[H]
	\center
	\begin{tabular}{lll}
	\toprule
	\textbf{Nationality} & \textbf{\ita{Nacionalidad}} & \textbf{Phonetic Pronunciation} \\
	\midrule
	English & \ita{ingl\'es} & Uh-meh-rih-kehn \\
	Spanish & \ita{espa\~nol} & Spa*n-ihsh \\
	Portuguese & \ita{portugués} & Por-chuh-geez \\
	French & \ita{francés} & Frehnch \\
	Russian & \ita{ruso} & Ruh-shee-ehn \\
	Hindi & \ita{hindi} & Hihn-dee \\
	Korean & \ita{coreano} & Koh-ree-ehn \\
	Chinese & \ita{chino} & Chah-ee-nees\\
	\bottomrule
	\end{tabular}
	\caption{Languages}
\end{table}

\section{Sobre la pronunciaci\'on}

\begin{itemize}
	\item In this case, \textbf{a*} indicates the ``a'' in ``apple.''
	\item Casi siempre se pronuncia la «h» en inglés;
			se pronuncia como una «j» en espa\~nol
	\item En inglés, la «z» no se pronuncia igual que una «s».
	\item «ah» = la «a» en espa\~nol
	\item «ee» = la «i» en espa\~nol
	\item «oo» = la «u» en español
	\item «»
\end{itemize}
