\chapter{Likes and Dislikes}

En este capítulo, vamos a aprender a expresar lo que nos gusta y lo que no nos gusta. \\

Unas tablas útiles:

\begin{table}[H]
	\centering
	\begin{tabular}{ll}
		\toprule
		\textbf{Frase} & \textbf{Traducción} \\
		\midrule
		I like \_\_\_\_ & \ita{Me gusta(n) \_\_\_\_} \\
		You like \_\_\_\_ & \ita{A ti te/A Ud. le/A Uds. les gusta(n) \_\_\_\_} \\
		He likes \_\_\_\_ & \ita{A él le gusta(n) \_\_\_\_} \\
		She likes \_\_\_\_ & \ita{A ella le gusta(n) \_\_\_\_} \\
		We like \_\_\_\_ & \ita{Nos gusta(n) \_\_\_\_} \\
		They like \_\_\_\_ & \ita{A ell@s les gusta(n) \_\_\_\_} \\
		\bottomrule
	\end{tabular}
	\caption{Likes}
\end{table}

Recuerden usar el verbo auxiliar \ita{to do} con la negación:

\begin{table}[H]
	\centering
	\begin{tabular}{lll}
		\toprule
		\textbf{Frase larga} & \textbf{Frase corta} & \textbf{Traducción} \\
		\midrule
		I do not like \_\_\_\_ & I don't like \_\_\_\_ & \ita{No me gusta(n) \_\_\_\_} \\
		You do not like \_\_\_\_ & You don't like \_\_\_\_ & \ita{A ti no te/A Ud. no le/A Uds. no les gusta(n) \_\_\_\_} \\
		He does not like \_\_\_\_ & He doesn't like \_\_\_\_ & \ita{A él no le gusta(n) \_\_\_\_} \\
		She does not like \_\_\_\_ & She doesn't like \_\_\_\_& \ita{A ella le gusta(n) \_\_\_\_} \\
		We do not like \_\_\_\_ & We don't like \_\_\_\_ & \ita{No nos gusta(n) \_\_\_\_} \\
		They do not like \_\_\_\_ & They don't like \_\_\_\_ & \ita{A ell@s no les gusta(n) \_\_\_\_} \\
		\bottomrule
	\end{tabular}
	\caption{Dislikes}
\end{table}

Una lista de pasatiempos (\ita{hobbies}):
\begin{table}[H]
	\centering
	\begin{tabular}{lll}
	\toprule
		\textbf{Activity} & \textbf{\ita{Traducci\'on}} &\\
	\midrule
		to dance & \ita{bailar} \\
		to exercise & \ita{hacer ejercicio}  \\
		to fish & \ita{pescar} \\
		to play basketball & \ita{jugar al baloncesto} \\
		to play cards & \ita{jugar cartas} \\
		to swim & \ita{nadar} \\
		to cook & \ita{cocinar} \\
		to play the guitar & \ita{tocar la guitarra} \\
		to listen to music & \ita{escuchar música} \\
		to watch TV & \ita{ver la tele} \\
		to read magazines & \ita{leer revistas} \\
		to work in the garden & \ita{trabajar en el jardín} \\
		to go to the movies & \ita{ir al cine} \\
		to go online & \ita{entrar en línea} \\
		to shop & \ita{ir de compras} \\
		to travel & \ita{viajar} \\
		to visit friends & \ita{visitar a los amig@s} \\
		to volunteer & \ita{trabajar como voluntari@} \\
	\bottomrule
	\end{tabular}
	\caption{Palabras de vocabulario: activities}
\end{table}

