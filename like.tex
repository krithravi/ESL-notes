\chapter{Likes and Dislikes}

En este capítulo, vamos a aprender a expresar lo que nos gusta y lo que no nos gusta. \\

Unas tablas útiles:

\begin{table}[H]
	\centering
	\begin{tabular}{ll}
		\toprule
		\textbf{Frase} & \textbf{Traducción} \\
		\midrule
		I like \_\_\_\_ & \emph{Me gusta(n) \_\_\_\_} \\
		You like \_\_\_\_ & \emph{A ti te/A Ud. le/A Uds. les gusta(n) \_\_\_\_} \\
		He likes \_\_\_\_ & \emph{A él le gusta(n) \_\_\_\_} \\
		She likes \_\_\_\_ & \emph{A ella le gusta(n) \_\_\_\_} \\
		We like \_\_\_\_ & \emph{Nos gusta(n) \_\_\_\_} \\
		They like \_\_\_\_ & \emph{A ell@s les gusta(n) \_\_\_\_} \\
		\bottomrule
	\end{tabular}
	\caption{Likes}
\end{table}

Recuerden usar el verbo auxiliar \emph{to do} con la negación:

\begin{table}[H]
	\centering
	\begin{tabular}{lll}
		\toprule
		\textbf{Frase larga} & \textbf{Frase corta} & \textbf{Traducción} \\
		\midrule
		I do not like \_\_\_\_ & I don't like \_\_\_\_ & \emph{No me gusta(n) \_\_\_\_} \\
		You do not like \_\_\_\_ & You don't like \_\_\_\_ & \emph{A ti no te/A Ud. no le/A Uds. no les gusta(n) \_\_\_\_} \\
		He does not like \_\_\_\_ & He doesn't like \_\_\_\_ & \emph{A él no le gusta(n) \_\_\_\_} \\
		She does not like \_\_\_\_ & She doesn't like \_\_\_\_& \emph{A ella le gusta(n) \_\_\_\_} \\
		We do not like \_\_\_\_ & We don't like \_\_\_\_ & \emph{No nos gusta(n) \_\_\_\_} \\
		They do not like \_\_\_\_ & They don't like \_\_\_\_ & \emph{A ell@s no les gusta(n) \_\_\_\_} \\
		\bottomrule
	\end{tabular}
	\caption{Dislikes}
\end{table}

Una lista de pasatiempos (\emph{hobbies}):
\begin{table}[H]
	\centering
	\begin{tabular}{lll}
	\toprule
		\textbf{Activity} & \textbf{\emph{Traducci\'on}} &\\
	\midrule
		to dance & \emph{bailar} \\
		to exercise & \emph{hacer ejercicio}  \\
		to fish & \emph{pescar} \\
		to play basketball & \emph{jugar al baloncesto} \\
		to play cards & \emph{jugar cartas} \\
		to swim & \emph{nadar} \\
		to cook & \emph{cocinar} \\
		to play the guitar & \emph{tocar la guitarra} \\
		to listen to music & \emph{escuchar música} \\
		to watch TV & \emph{ver la tele} \\
		to read magazines & \emph{leer revistas} \\
		to work in the garden & \emph{trabajar en el jardín} \\
		to go to the movies & \emph{ir al cine} \\
		to go online & \emph{entrar en línea} \\
		to shop & \emph{ir de compras} \\
		to travel & \emph{viajar} \\
		to visit friends & \emph{visitar a los amig@s} \\
		to volunteer & \emph{trabajar como voluntari@} \\
	\bottomrule
	\end{tabular}
	\caption{Palabras de vocabulario: activities}
\end{table}

En las tablas que se pueden ver arriba, hay varios espacios.
Esos espacios se pueden llenar con el infinitivo o el gerundio, o simplemente un sustantivo. \\

Ahora vamos a aprender a formar los gerundios
\section{Gerunds}

Pueden seguir las siguientes reglas para formar los gerundios:
\begin{enumerate}[noitemsep]
	\item Comenzar con el infinitivo, como \ita{eat} (comer) o \ita{write} (escribir)
	\item Si el infinitivo termina con \ita{-e}, pero no \ita{-ee}, quítenlo.
	\item Si el infinitivo termina con \ita{-ie}, reemplácenlo con \ita{-y}
	\item Si el infinitivo termina con \ita{consonante-vocal-consonante}, doblen la última consonante
	\item Agreguen el sufijo \ita{-ing}
\end{enumerate}

Vamos a ver unos ejemplos:\\



\textbf{Put $\rightarrow$ Putting}\\
\begin{itemize}[noitemsep]
	\item ``put'' (poner) es el infinitivo
	\item El infitivo termina con \ita{consonante-vocal-consonante}. Por eso, hay que doblar la última consonante. Ahora, tenemos ``putt''
	\item Hay que agregar el sufijo \ita{-ing} para llegar al gerundio ``putting.''
\end{itemize}

\section{Ejemplos}
Las frases siguientes son las que cubrimos en clase:

