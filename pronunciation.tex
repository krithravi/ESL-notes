\chapter{README!/\textexclamdown LÉANME!}

\section{Pronunciation}
Por favor, lean este cap\'itulo antes de seguir con el resto de la gu\'ia.
En esta gu\'a, voy a indicar la pronunciaci\'on de las palabras de manera fon\'etica.
He puesto aquí la notaci\'on que voy a utilizar.


\begin{itemize}
	\item \textbf{a*} indicates the ``a'' in ``apple.''
	\item Casi siempre se pronuncia la «h» en inglés;
			se pronuncia como una «j» en espa\~nol
	\item En inglés, la «z» no se pronuncia igual que una «s».
	\item «ah» = la «a» en espa\~nol
	\item «ee» = la «i» en espa\~nol
	\item «oo» = la «u» en espa\~nol
	\item «oh» = la «o» en espa\~nol
	\item «ey» = la «e» en espa\~nol
	\item «uh» = \ita{schwa} (lo que cubrimos en clase)
\end{itemize}

\begin{conf}{Acerca de la «z»}
Quiero que hagan lo siguiente: toquen su garganta, y pronuncien una «v» y luego una «f».
	Ahora, intenten la misma cosa con «b» y «p».\\
\\
\textquestiondown Notaron algo? Con la «v» y la «b», deber\'ian haber
	sentido una vibraci\'on, pero no con la «f» ni la «p». \\
\\
En la ling\"u\'istica, las consonantes con las cuales se puede sentir esa vibraci\'on
	se llaman \textbf{consonantes sonoras}, las dem\'as son \textbf{sordas}. Suelen
	aparecer en parejas. Por ejemplo, la \'unica diferencia entre una «b» y una «p» es
	que la «b» es una consonante sonora y la «p» es una consonante sorda.\\
\\
En ingl\'es, la «z» es una consonante sonora, y la «z» y la «s» son una pareja como las que se mencionaron antes. Deben poder sentir esa vibraci\'on en la garganta.
\end{conf}

\section{Otras cosas}

Suelo poner cosas que se pueden reemplazar entre ``<'' y ``>''.
Por ejemplo, en español, una forma de indicar obligación en español es usar
la fórmula \ita{tener que <infinitivo>}.
