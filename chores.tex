\chapter{Chores}

A bit of vocab!

\begin{table}[H]
	\centering
	\begin{tabular}{lll}
	\toprule
		\textbf{Chores} & \textbf{\ita{Traducci\'on}}\\
	\midrule
		to wash the dishes & \ita{lavar los platos}\\
		to dry the dishes & \ita{secar los platos}\\
		to do the laundry & \ita{lavar la ropa} \\
		to make lunch & \ita{preparar el almuerzo}\\
		to make the bed & \ita{arreglar la cama} \\
		to do homework & \ita{hacer la tarea} \\
		to take care of the children & \ita{cuidar a los ni\~nos} \\
		to cut the grass & \ita{cortar la hierba} \\
		to clean the car & \ita{limpiar el coche} \\
		to take out the trash & \ita{tirar la basura} \\
		to get/pick up the mail & \ita{recoger el correo} \\
		to wash the car & \ita{lavar el carro} \\
		to water the plants & \ita{regar las plantas} \\
		to walk the dog & \ita{pasear al perro} \\
		to feed the dog & \ita{alimentar al perro} \\
	\bottomrule
	\end{tabular}
	\caption{Chores}
\end{table}

\section{Los verbos}%
\label{sec:Los verbos}

\begin{conf}{Verbos: Infinitivos/Gerundios}
	\begin{itemize}
		\item El infinitivo es la forma más sencilla de un verbo.
			En español, terminan con \ita{-ar}, \ita{-er}, o \ita{-ir}
			(como «saltar», «caer», y «sentir»). En inglés, empiezan con «to»
			(como «to jump», «to fall» o «to feel»).
		\item Después de una preposición, siempre se usa el infinitivo
		\item Después de un verbo conjugado, se puede usar:
			\begin{itemize}
				\item el infinitivo: \ita{Quiero \un{dormir}} \arr \ita{I want \un{to sleep}.}
				\item un participio pasado: \ita{He \un{comido}} \arr \ita{I have \un{eaten}}.
					En español, estas formas siguen una conjugación del verbo «haber».
					En inglés, siguen el verbo «to have».
				\item un gerundio: \ita{Ella está \un{durmiendo}} \arr \ita{She is \un{sleeping}.}
					En español, estas formas siguen una conjugación del verbo «estar».
					En inglés, siguen el verbo «to be».
			\end{itemize}
		\item No se usa ``to'' con los gerundios.
	\end{itemize}
\end{conf}

Unos ejemplos para que puedan ver las diferencias entre los significados y la gramática.
\begin{enumerate}
	\item \ita{hacer la tarea} (to do homework)
		\begin{itemize}
			\item \ita{Quiero hacer la tarea.} \arr I want to do homework.
			\item \ita{Hago la tarea.} \arr I do the homework.
			\item \ita{Estoy haciendo la tarea.} \arr I am doing homework.
		\end{itemize}
	\item \ita{limpiar el coche} (to clean the car)
		\begin{itemize}
			\item \ita{Necesitas limpiar el coche.} \arr You need to clean the car.
			\item \ita{Limpias el coche} \arr You clean the car.
			\item \ita{Estás limpiando el coche} \arr You are cleaning the car.
		\end{itemize}
	\item \ita{preparar el almuerzo} (to make lunch)
		\begin{itemize}
			\item \ita{Voy a preparar el almuerzo} \arr I am going to make lunch.
			\item \ita{Preparo el almuerzo.} \arr I make lunch.
			\item \ita{Estoy preparando el almuerzo} \arr I am making lunch.
		\end{itemize}
\end{enumerate}

\section{Más práctica}%
\label{sec:Más práctica}

\begin{itemize}
	\item \ita{Estoy recogiendo el correo} \arr I am getting the mail.
	\item \ita{Voy a recoger el correo} \arr I am going to get the mail.
	\item \ita{Voy a pasear al perro} \arr I am going to walk the dog.
	\item \ita{Vas a pasear al perro} \arr You are going to walk the dog.
	\item \ita{Él pasea al perro} \arr He walks the dog.
	\item \ita{Ellos alimentan al perro} \arr They feed the dog.
	\item \ita{Estoy alimentando al perro} \arr I am feeding the dog.
	\item \ita{Yo cuido a los niños} \arr I take care of the children.
	\item \ita{Estoy cuidando a los niños} \arr I am taking care of the children.
	\item \ita{Ella cuida a los niños} \arr She takes care of the children.
	\item \ita{Nosotros cuidamos a los niños} \arr We take care of the children.
	\item \ita{Nosotros estamos cuidando a los niños} \arr We are taking care of the children.
	\item \ita{Ellos cuidan a los niños} \arr They take care of the children.
	\item \ita{Yo lavo la ropa} \arr I do the laundry.
	\item \ita{Estoy lavando la ropa} \arr I am doing the laundry.
	\item \ita{Tengo que lavar la ropa} \arr I have to do the laundry.
	\item \ita{Voy a lavar la ropa} \arr I am going to do the laundry.
	\item \ita{Estoy lavando el coche} \arr I am washing the car.
\end{itemize}

Preguntas y respuestas:
\begin{itemize}
	\item What \underline{are} they doing? \arr \ita{\textquestiondown Qué están haciendo?}
	\item They are making lunch \arr \ita{Están preparando el almuerzo.}
	\item What \underline{is} he doing? \arr \ita{\textquestiondown Qué está haciendo él?}
	\item He is washing the dishes. \arr \ita{Él está lavando los platos.}
	\item They are making the bed. \arr \ita{Ellos están arreglando la cama.}
	\item He is taking out the trash. \arr \ita{Él está sacando la basura.}
	\item He is washing the car. \arr \ita{Él está lavando el coche.}
	\item She is watering the grass. \arr \ita{Ella está regando la hierba.}
\end{itemize}
