\chapter{Possesive Pronouns}

\begin{itemize}
	\item Los pronombres posesivos indican que alguien posee algo.
\end{itemize}

\begin{table}[H]
	\centering
	\begin{tabular}{lll}
	\toprule
	\textbf{Pronoun} & \textbf{\ita{Significado}} & \textbf{Pronunciation} \\
	\midrule
	My & \ita{mi/de m\'i} & mah-ee \\
	Your & \ita{tu/de ti, su/de Ud(s).} & yor \\
	Our & \ita{nuestro/de nosotros} & ah-wer\\
	His & \ita{su/de él} & hihz \\
	Her & \ita{su/de ella} & huhr \\
	Their & \ita{su/de ellos} & dheyr \\
	\bottomrule
	\end{tabular}
	\caption{Possesive pronouns}
\end{table}

\begin{conf}{Pronombre neutral de g\'enero}
A veces se usa la forma « their » con sujetos singulares; es un pronombre neutral
de g\'enero.
Por ejemplo, se puede decir, \ita{I don't know their email} («No sé su correo»).
Esta forma indica que la persona usa los pronombres \ita{they/their}, o que
no sabemos su g\'enero.
\end{conf}

\section{Examples}

Vamos a ver unos ejemplos en las frases siguientes y sus traducciones:

\begin{itemize}
	\item Martín tiene un lápiz. Su lápiz es verde. 
\end{itemize}
