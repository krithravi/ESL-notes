\chapter{Possesion}

\section{Pronombres posesivos}
\begin{itemize}
	\item Los adjetivos posesivos indican que alguien posee algo.
\end{itemize}

\begin{table}[H]
	\centering
	\begin{tabular}{lll}
	\toprule
	\textbf{Pronoun} & \textbf{\ita{Significado}} & \textbf{Pronunciation} \\
	\midrule
	My & \ita{mi/de m\'i} & mah-ee \\
	Your & \ita{tu/de ti, su/de Ud(s).} & yor \\
	His & \ita{su/de él} & hihz \\
	Her & \ita{su/de ella} & huhr \\
	Our & \ita{nuestro/de nosotros} & ah-wer\\
	Their & \ita{su/de ellos} & dheyr \\
	\bottomrule
	\end{tabular}
	\caption{Possesive pronouns}
\end{table}

\begin{conf}{\textexclamdown Ojo!}
	Se puede ver que en espa\~nol, se usa «su» para indicar posesi\'on
	en varios casos, pero en ingl\'es, hay más distinctiones. Por eso,
	cuidado con el g\'enero y el n\'umero de personas.
\end{conf}

\subsection{Examples}

Vamos a ver unos ejemplos en las frases siguientes y sus traducciones:

\begin{enumerate}
	\item Usted tiene un lápiz. Su l\'apiz es verde.
		\arr \ita{You have a pencil. Your pencil is green.}
	\item Manuela tiene un lagarto. Su lagarto es amarillo.
		\arr \ita{Manuela has a lizard. Her lizard is yellow.}
	\item Ricardo Arjona es un cantante talentoso. Sus canciones son muy rom\'anticas.
		\arr \ita{Ricardo Arjona is a talented singer. His songs are very romantic.}
	\item Arjona y Chayanne cantan baladas. Sus canciones son hermosas.
		\arr \ita{Arjona and Chayanne sing ballads. Their songs are beautiful.}
\end{enumerate}

En estos ejemplos, la palabra «su» se reemplaza
por \ita{your}, \ita{his}, \ita{her}, o \ita{their}.

\section{Pronombres posesivos}

Hay una distinci\'on entre los pronombres posesivos y los adjetivos posesivos.
Esta distinci\'on es m\'as f\'acil de explicar con unos ejemplos:

\begin{itemize}
	\item Este l\'apiz es m\'io.
		\arr \ita{This pencil is mine.}
	\item Mi lagarto es amarillo, pero el tuyo es rojo.
		\arr \ita{My lizard is yellow, but yours is red.}
\end{itemize}

Estos pronombres se usan en un contexto diferente, como se puede ver.
Vean esta tabla:

\begin{table}[H]
	\centering
	\begin{tabular}{lll}
	\toprule
	\textbf{El sujeto} & \textbf{Adjetivo posesivo} & \textbf{Pronombre posesivo} \\
	\midrule
	I & my & mine \\
	You & your & yours \\
	He & his & his \\
	She & her & her \\
	It & its & -- \\
	We & our & ours\\
	They & their & theirs \\
	\bottomrule
	\end{tabular}
	\caption{Possesive pronouns}
\end{table}


\section{Pronombre neutral de g\'enero}
\begin{conf}{Pronombre neutral de g\'enero}
\textquestiondown Qu\'e se hace cuando no sabemos el g\'enero de la persona de
la cual estamos hablando? \\
\\
A veces se usa la forma « their » con sujetos singulares; es un pronombre neutral
de g\'enero.
Por ejemplo, se puede decir, \ita{I don't know their email} («No sé su correo»).
Esta forma indica que la persona usa los pronombres \ita{they/their}, o que
no sabemos su g\'enero.
\end{conf}

