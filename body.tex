\chapter{Body}

\section{The doctor's office!}

M\'as vocabulario; espero que est\'en listos :)

\begin{table}[H]
	\centering
	\begin{tabular}{lll}
	\toprule
		\textbf{Noun} & \textbf{\ita{Sustantivo}} & \textbf{Phonetic Pronunciation}\\
	\midrule
		doctor & \ita{doctor} & dahk-tuhr \\
		nurse & \ita{enfermer@} & nuhrs \\
		patient & \ita{paciente/enferm@} & pey-shehnt \\
		medicine & \ita{medicina} & meh-dih-sihn \\
		pills & \ita{p\'ildoras} & pihlz \\
		stomachache & \ita{dolor de est\'omago} & stuh-muhk-eyk \\
		headache & \ita{dolor de cabeza} & hehd-eyk \\
		toothache & \ita{dolor de muelas} & tooth-eyk \\
		sore throat & \ita{dolor de garganta} & sohr throht \\
		fever & \ita{fiebre} & fee-vehr \\
		cold & \ita{refriado} & cohld \\
	\bottomrule
	\end{tabular}
	\caption{Doctor's office-related vocabulary!}
\end{table}

\begin{table}[H]
	\centering
	\begin{tabular}{lll}
	\toprule
		\textbf{Noun} & \textbf{\ita{Sustantivo}} & \textbf{Phonetic Pronunciation}\\
	\midrule
		foot & \ita{pie} & foot \\
		feet & \ita{pies} & feet \\
		leg & \ita{pierna} & lehg \\
		toe & \ita{dedo de pie} & toh \\
		hand & \ita{mano} & ha*nd \\
		arm & \ita{brazo} & ahrm \\
		finger & \ita{dedo} & fihn-gehr \\
		tooth & \ita{diente} & tooth \\
		teeth & \ita{dientes} & teeth \\
	\bottomrule
	\end{tabular}
	\caption{Body-related vocabulary!}
\end{table}

Tambi\'en, les dejo una lista de palabras que cubrimos en clase que no tienen que saber.
\begin{table}[H]
	\centering
	\begin{tabular}{lll}
	\toprule
		\textbf{Noun} & \textbf{\ita{Sustantivo}} & \textbf{Phonetic Pronunciation}\\
	\midrule
		magazine & \ita{revista} & ma*-guh-zeen \\
		waiting room & \ita{sala de espera} & wey-tihng room \\
		toddler & \ita{ni\~n@ de 2 o 3 a\~nos} & tahd-lehr \\
		neck & \ita{cuello} & nehk \\
		button & \ita{bot\'on} & buht-ton \\
		horn & \ita{cuerno} & hohrn \\
		appointment & \ita{cita} & uh-poh-ihnt-mehnt \\
		paperwork & \ita{papeleo} & pey-puhr-wehrk \\
		insurance & \ita{seguranza} & in-shoor-ehns \\
		card & \ita{tarjeta} & cahrd\\
		blood pressure & \ita{presi\'on arterial} & bluhd prehshuhr \\
		temperature & \ita{temperatura} & tehmp-ehr-ah-chuhr \\
	\bottomrule
	\end{tabular}
	\caption{Optional vocabulary}
\end{table}

\section{What hurts?}

La frase «\textquestiondown Qu\'e te/le duele?» se puede traducir
como \ita{What hurts?}. \\

Imag\'inense una conversaci\'on entre dos personas, Jack y Jill: \\

$\quad$ -- Jack: What hurts?

$\quad$ -- Jill: \_\_\_\_\_\_\_\_

$\quad$ -- Jack: \_\_\_\_\_\_\_\_ \\

\textquestiondown Con qu\'e puede contestar Jill?\\

Ella va a utilizar la forma «\ita{my}» + «parte del cuerpo» + «hurt(s)». \\

\textquestiondown C\'omo se sabe cu\'al forma hay que utilizar, \ita{hurt} o \ita{hurts}?\\

Si el sujeto es singular, se usa \ita{hurts}. Si el sujeto es plural, se usa \ita{hurt}.
Por ejemplo, con \ita{leg} (pierna), se usa \ita{hurts}, pero con \ita{arms} (brazos), se usa \ita{hurt}. \\

Por lo tanto, la conversaci\'on puede suceder as\'i:\\

$\quad$ -- Jack: What hurts?

$\quad$ -- Jill:  My leg hurts./My arms hurt. (\ita{Me duele la pierna./Me duelen los brazos.})

$\quad$ -- Jack: I'm sorry. \\


\section{Unas enfermedades/unos s\'intomas comunes}

Para expresar que alguien tiene una enfermedad o que experimenta un s\'intoma,
se usa la f\'ormula <pronombre personal/nombre de la persona> +
<conjugaci\'on del verbo \ita{to have}> +
<\ita{a/an}> + <la enfermedad/el s\'intoma>. \\

Primero, repasemos la conjugaci\'on el verbo \ita{to have} (tener).

\begin{table}[H]
	\centering
	\begin{tabular}{ll}
	\toprule
		\textbf{Pronoun} & \textbf{Conjugation} \\
	\midrule
		I & have\\
		You & have \\
		She/He & has \\
		We & have \\
		They & have \\
	\bottomrule
	\end{tabular}
	\caption{``to have'' Conjugation}
\end{table}

Veamos unos ejemplos:

\begin{itemize}
	\item Ella tiene dolor de cabeza.
		\arr \ita{She has a headache.}
	\item Tengo una fiebre.
		\arr \ita{I have a fever}.
	\item Iv\'an tiene dolor de garganta.
		\arr \ita{Iv\'an has a sore throat.}
	\item Uma est\'a resfriada.
		\arr \ita{Uma has a cold.}
	\item David tiene dolor de est\'omago.
		\arr \ita{David has a stomachache.}
\end{itemize}

\begin{conf}{Hot and Cold}

Para decir que alguien \ita{tiene} fr\'io/calor, se usa la
expresi\'on \ita{to be cold/hot}.\\ \\

Les dejo la conjugaci\'on del verbo \ita{to be} como referencia:\\

\begin{table}[H]
	\centering
	\begin{tabular}{lll}
	\toprule
		\textbf{Pronoun} & \textbf{Conjugation} & \textbf{Contraction}\\
	\midrule
		I & am & I'm\\
		You & are & You're \\
		She/He/It & is & She's/ He's\\
		We & are & We're \\
		They & are & They're \\
	\bottomrule
	\end{tabular}
	\caption{``to be'' Conjugation}
\end{table}

Vamos a ver unos ejemplos:
	\begin{itemize}
		\item Tenemos fr\'io. \arr \ita{We are cold/We're cold}.
		\item Miguel tiene fr\'io. \arr \ita{Miguel is cold}.
		\item Tengo calor. \arr \ita{I am hot./I'm hot}.
	\end{itemize}

Para decir que alguien est\'a resfriad@, se usa la expresi\'on
\ita{to have a cold}.\\

\textexclamdown No se confundan con las dos expresiones!
\end{conf}

\section{Unas frases \'utiles}

\begin{itemize}
	\item \textquestiondown Est\'as bien? \arr \ita{Are you okay?}
	\item \textquestiondown C\'omo te sientes? \arr \ita{How are you feeling?}
	\item \textquestiondown Qu\'e te pasa? \arr \ita{What's the matter?}
	\item \textquestiondown Ya tomaste algo? \arr \ita{Did you take something for it?}
	\item \textquestiondown Puedo hacer algo por ti? \arr \ita{Can I do something for you?}
	\item \textquestiondown C\'omo te puedo ayudar? \arr \ita{How can I help you?}
	\item \textquestiondown Te puedo ayudar? \arr \ita{Can I help you?}
	\item \textquestiondown Necesitas ayuda? \arr \ita{Do you need help?}
	\item \textquestiondown Puedes ayudarme? \arr \ita{Can you help me?}
\end{itemize}

Una conversaci\'on:

$\quad$ -- Jack: What's the matter?

$\quad$ -- Jill: I have a cold. (\ita{Estoy resfriada.})

$\quad$ -- Jack: Did you take something for it? (\ita{\textquestiondown Te tomaste algo?})

$\quad$ -- Jill: I did/I did not. (\ita{S\'i, tom\'e algo. / No, no tom\'e nada.})
