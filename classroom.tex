\chapter{Office/classroom supplies}

\section{Vocabulario}
Comencemos con el vocabulario.

\begin{table}[H]
	\centering
	\begin{tabular}{lll}
	\toprule
	\textbf{Object} & \textbf{\ita{Objeto}} & \textbf{Phonetic Pronunciation}\\
	\midrule
	desk & \ita{escritorio} & dehsk\\
	table & \ita{mesa} & teybl \\
	dictionary & \ita{diccionario} & dihk-shun-eh-ree \\
	notebook & \ita{cuaderno} & noht-book \\
	eraser & \ita{borrador} & ee-rey-zehr \\
	backpack & \ita{mochila} & ba*k - pa*ck\\
	pencil & \ita{l\'apiz} & pehn-cihl \\
	book & \ita{libro} & book \\
	whiteboard & \ita{pizarra} & wah-eet-boh-rd \\
	pen & \ita{bol\'igrafo} & pehn \\
	ruler & \ita{regla} &  roo-lehr \\
	computer & \ita{computadora} & cuhm-pyoo-tehr \\
	paper & \ita{papel} & pey-pehr \\
	\bottomrule
	\end{tabular}
	\caption{Palabras requisitas}
\end{table}

\begin{table}[H]
	\centering
	\begin{tabular}{lll}
	\toprule
	\textbf{Object} & \textbf{\ita{Objeto}} & \textbf{Phonetic Pronunciation}\\
	\midrule
	filing cabinet & \ita{archivador} & fah-ee-lihng ca*-bih-neht \\
	stapler & \ita{grapadora} & stey-plehr \\
	clock & \ita{reloj} & clah-ck \\
	laptop & \ita{computadora p\'ortatil} & la*p-tahp \\
	(tele)phone & \ita{m\'ovil} & (teh-leh)fohn \\
	marker & \ita{marcador} & mahrk-ehr\\
	highlighter & \ita{resaltador} & hah-ee-lah-ee-tehr \\
	scissors & \ita{tijeras} & sih-sehrs \\
	mouse & \ita{rat\'on} & ma*-oos \\
	headphones & \ita{aud\'ifonos} & hehd-fohns \\
	glue & \ita{goma} & gloo\\
	recycling bin & \ita{papelera} & ree-sah-ee-clihng bihn \\
	cardboard & \ita{cart\'on} & cahrd-bohrd \\
	poster & \ita{cartel} & poh-stehr \\
	sharpener & \ita{sacapuntas} & shahr-pehn-ehr \\
	tape & \ita{cinta} & teyp \\
	drawer & \ita{caj\'on/gaveta} & dror \\
	key & \ita{llave} & kee\\
	box & \ita{caja} & bah-ks \\
	\bottomrule
	\end{tabular}
	\caption{Palabras opcionales}
\end{table}

\section{A y An}

\textquestiondown Cu\'al es la diferencia entre los
art\'iculos indefinidos \ita{a} y \ita{an}?\\

Se usa «a» enfrente de palabras que comienzan con un sonido de consonante,
y «an» con palabras que comienzan con un sonido de vocal.\\

Y esto, \textquestiondown qu\'e significa? Vamos a ver con unos ejemplos.

\begin{table}[H]
	\centering
	\begin{tabular}{ll}
	\toprule
	\textbf{«A»} & \textbf{«An»} \\
	\midrule
	a frog & an apple \\
	a lizard & an eraser \\
	a table & an axe \\
	a tomato & an ear \\
	a hack & an hour \\
	a university & an understanding \\
	\bottomrule
	\end{tabular}
	\caption{Unos ejemplos con los art\'iculos indefinidos}
\end{table}

Creo que las primeras 4 l\'ineas son bastante f\'aciles de entender.
Pero, las palabras \ita{hack} y \ita{hour} comienzan con una «h», una
consonante. \textquestiondown Por qu\'e la primera se usa con \ita{a}
y la segunda con \ita{an}? \\

Es porque la «h» se pronuncia en \ita{hack}, pero no en \ita{hour}.
Debido a que la primera palabra comienza en un sonido de consonante - un
sonido de «h» - se usa con \ita{a}. La «h» es muda en «hora», y esto
hace que la palabra comience con el sonido de «a», que es un sonido de vocal.
Por eso, se dice \ita{an hour}.\\

Se puede ver el mismo fen\'omeno con \ita{a university} (una universidad) y
\ita{an understanding} (un entendimiento). La palabra \ita{university} empieza
con un sonido de «y»; la primera s\'ilaba se pronuncia como «yu». Ya que
comienza con un sonido de consonante, se usa con \ita{a}. Sin embargo,
\ita{understanding} comienza con un sonido de «uh», que es un sonido de vocal.
Por lo tanto, se usa con \ita{an}.\\

Esto tambi\'en sucede con los adjetivos, no solo con los sustantivos.

\begin{table}[H]
	\centering
	\begin{tabular}{ll}
	\toprule
	\textbf{«A»} & \textbf{«An»} \\
	\midrule
	a great example & an enormous jellyfish \\
	a tiny frog & an animated film \\
	a cute lizard & an interesting person \\
	a holy place & an organized desk\\
	a tall tree & an adorable squid \\
	\bottomrule
	\end{tabular}
	\caption{Unos ejemplos con los art\'iculos indefinidos}
\end{table}

\section{Prepositions!}

Las preposiciones nos permiten crear frases m\'as complejas e interesantes
con las palabras que sabemos. \\

Unas preposiciones: \footnote{\textquestiondown Pueden ver que me encantan las tablas?}

\begin{table}[H]
	\centering
	\begin{tabular}{ll}
	\toprule
	\textbf{Preposition} & \textbf{Preposiciones} \\
	\midrule
	in/inside of & \ita{dentro de} \\
	on/on top of & \ita{sobre} \\
	above & \ita{encima de} \\
	under/below & \ita{debajo de} \\
	in front of & \ita{delante de} \\
	behind & \ita{detr\'as de}\\
	to/on the left of & \ita{a la izquierda de} \\
	to/on the right of & \ita{a la derecha de} \\
	\bottomrule
	\end{tabular}
	\caption{Unos ejemplos con los art\'iculos indefinidos}
\end{table}

Ahora que hemos aprendido unas preposiciones,
vamos a hacer ver unas frases con sus traducciones correspondientes.

\begin{itemize}
	\item El l\'apiz est\'a sobre mi escritorio.
		\arr \ita{The pencil is on my desk.}
	\item Mi papelera est\'a debajo de mi escritorio.
		\arr \ita{My recycling bin is below/under my desk.}
	\item El borrador est\'a delante de la pizarra.
		\arr \ita{The eraser is in front of the whiteboard.}
	\item Una regla est\'a sobre el libro.
		\arr \ita{A ruler is on top of the book.}
	\item El cartel est\'a encima de mi computadora.
		\arr \ita{The poster is above my computer.}
	\item Mi tel\'efono est\'a a la izquierda de la mesa.
		\arr \ita{My phone is on the left of the table.}
	\item Mis llaves est\'an debajo de la mesa.
		\arr \ita{My keys are under the table.}
	\item Se me olvidaron las llaves dentro del auto.
		= Olvid\'e mis llaves dentro del auto.
		\arr \ita{I forgot my keys in/inside the car.}
	\item The glue is in the box.
		\arr \ita{La goma est\'a dentro de la caja.}
\end{itemize}


Otras cosas relacionadas. Hay m\'as preposiciones que pueden ser \'utiles
para Uds.

\begin{conf}{Sobre «sobre»}

\end{conf}
